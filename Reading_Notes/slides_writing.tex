\documentclass[notes=show,beamer,compress]{beamer}
\usepackage{amsmath}
\usepackage{amssymb}
\usepackage{graphicx}
\usepackage{mathpazo}
\usepackage{hyperref}
\usepackage{multimedia}
%\everymath{\displaystyle}
\setbeamertemplate{footline}[frame number] 
\setbeamertemplate{caption}[numbered] 
\usepackage[export]{adjustbox}

\usetheme{Madrid}
\usecolortheme{beaver}
\begin{document}
\begin{frame}
\title{Writing}
\date{}
\titlepage
\end{frame}

\section*{}

\begin{frame}\frametitle{About the paper}
BOARD
\end{frame}


\begin{frame}\frametitle{}
\begin{quote}
Nevertheless, in my own view, this paper contains at least two logical fallacies when doing its analysis.
\end{quote}
\end{frame}

\begin{frame}\frametitle{}
\begin{quote}
 In this paper, the authors show the gender difference of career outcomes. First, they show that
the gender gaps exist persistently in recent years. Second, they investigate what causes the gender
differences with some different performance measures. In this step, they got two take-aways: (i)
the effect of the existence of young kids in their home differs significantly between male and female
lawyers; and (ii) other explanations does not explain the gender gaps. Third, the authors also
investigate the relationship between gender gaps and performance in the employees’ career, but
other reasons do not contribute to such gaps. In conclusion, the authors find two takeaways: (i)
gender gaps exist persistently among the U.S. lawyers in recent years; and (ii) gender gaps in
earnings partly come from the career performance.
\end{quote}
\end{frame}

\begin{frame}\frametitle{}
\begin{quote}
This paper uses survey data from two years of 2002 and 2007 to examine whether there
are gender gaps in the legal profession. The author chooses billed hours and new client
revenue as two measures of performance..... Throughout the study,
the authors verified the significance of different variables by multiple linear regression. The
main idea of the paper is that gender gaps in earnings and promotion exist in the legal
profession and, aspirations and child-rearing both are the non-negligible determinants.
\end{quote}
\end{frame}

\begin{frame}\frametitle{}
\begin{quote}
 The authors examine the persistent performance gap between genders in
the legal profession and attempt to explain potential determinants of this gap. The ramifications of this performance gap can help potentially
explain the existence of both earnings and advancement gaps between genders. The authors also argue that their results can be extrapolated to
other high-skilled professions and help explain the overall persistent wage and advancment gaps between genders. 
\end{quote}
\end{frame}

\begin{frame}\frametitle{}
\begin{quote}
 This paper mainly used the method of linear regression to analyze the differences between male
and female lawyers' professional performance. Based on previous studies, authors proposed some
hypotheses that may influence the results. After that, some variables were defined according to
these hypotheses. Finally, regression analysis was conducted to verify whether these assumptions
had an impact on male and female performance differences.
\end{quote}
\end{frame}

\begin{frame}\frametitle{}
\begin{quote}
 The authors use data from After the JD, a national, longitudinal survey of lawyers
in the United States to study the gender gap in the performance of lawyers in the US
and the determinants. Azmat and Ferrer used two methods to measure gender gap in
the performance. One is the Hours Billed, and the other is the New Client Revenue
\end{quote}
\end{frame}

\begin{frame}\frametitle{}
\begin{quote}
In the article “Gender Gaps in Performance: Evidence from Young Lawyers”, Ghazala Azmat and
Rosa Ferrer (2017) focused on two problems related to gender gaps in performance. One is
what determinants cause the gender gaps existing in performances. The other is how gender
gaps in performances influence career outcomes. In order to measure lawyers’ performance,
the authors introduce two methods: hours billed and new client revenue which are two
fundamental variables in authors’ following models.
\end{quote}
\end{frame}

\begin{frame}\frametitle{}
https://econ.duke.edu/masters-programs/current-students/resources/writing-resources
\end{frame}



\end{document}

